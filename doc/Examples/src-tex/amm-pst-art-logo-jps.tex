%===============================================================================
% File Name     : <demo-art-logo.tex>
% Description   : User Guide or LaTeX package <pst-art-logo.sty>
%-------------------------------------------------------------------------------
% Author        : Amit Manohar Manthanwar
% Mailer        : manthanwar@hotmail.com
% Mobile        : +91.853.081.3398
% WebURL        : https://manthanwar.github.io
%-------------------------------------------------------------------------------
% Copyright     : ©2024 Amit Manohar Manthanwar
% License       : LaTeX Project Public License
%-------------------------------------------------------------------------------
% This program can be redistributed and/or modified under the terms
% of the LaTeX Project Public License Distributed from CTAN archives
% in directory macros/latex/base/lppl.txt.
%===============================================================================
% Revision Log  | Author  | Description
%---------------+---------+----------------------------------------------------
% 25-Nov-2024   | AMM     | Initial Version
%---------------+---------+----------------------------------------------------
%===============================================================================
\documentclass{pst-icon}
\usepackage{pst-art-logo}
% \usepackage{FiraSans}
%-------------------------------------------------------------------------------
\title{Dynamic Logo Art Design Using \LaTeX}
\author{Amit M. Manthanwar}
%===============================================================================
\begin{document}
\maketitle
%===============================================================================
\begin{figure}[!h]
\centering%
\resizebox{0.6\textwidth}{!}{%\sffamily
% \psset{unit=1mm}
\begin{pspicture}(0,0)(4,4)
% \psgrid[subgriddiv=5, gridcolor=gray!50, subgridcolor=gray!20, gridlabels=4pt, gridlabelcolor=red, gridwidth=0.1mm,subgridwidth=0.1mm] % griddots=4, subgriddots=10,
%
\rput(0,0.2){\psLogoJapanLady[iconWidth=20]}%
%
\end{pspicture}}%
% \caption{text box label}%
% \label{box Caption Label}%
\end{figure}%===============================================================================
\section{Construction}
%===============================================================================
\begin{figure}[!h]
\centering%
\resizebox{0.6\textwidth}{!}{%\sffamily
% \psset{unit=1mm}
\begin{pspicture}(0,0)(4,4)
\psgrid[subgriddiv=5, gridcolor=gray!50, subgridcolor=gray!20, gridlabels=4pt, gridlabelcolor=red, gridwidth=0.1mm,subgridwidth=0.1mm] % griddots=4, subgriddots=10,
%
\rput(0,0.2){\psLogoJapanLady[iconWidth=20]}%
%
\newcommand{\abox}[1]{\psset{fillstyle=solid,fillcolor=black,linestyle=none,linewidth=0}\pscircle(0,0){0.1}\rput(0,0){\color{white}{\tiny{#1}}}}%
\rput(2.7,1.7){\abox{1}}%
\rput(1.3,2.9){\abox{2}}%
\rput(2.6,4.0){\abox{3}}%
\rput(2.0,2.15){\abox{4}}%
\rput(3.25,4.0){\abox{5}}%
\rput(0.55,0.0){\abox{6}}%
\rput(3.75,4.0){\abox{7}}%
\rput(2.6,0.7){\abox{8}}%
\rput(3.6,2.2){\abox{9}}%
%
% drawCircleBgM
% drawPathCircR1
%
\end{pspicture}}%
% \caption{text box label}%
% \label{box Caption Label}%
\end{figure}%===============================================================================
\rput[rc](15.5,6){\begin{minipage}{40mm}
\begin{enumerate}
\item drawCircleBgM
\item drawPathCircR1
\item drawPathLineB1
\item drawPathLineY1
\item drawPathLineR1
\item drawPathLineG1
\item drawPathLineG2
\item drawPathLineG3
\item drawPathLineR2
\end{enumerate}%
\end{minipage}}
%===============================================================================
\clearpage
% \hypersetup{
% colorlinks=true, % Ensures links are colored
% urlcolor=teal,   % Sets the color of URL links (adjust as desired)
% linkbordercolor=red,
% pdfborderstyle={/S/U/W 1} % Underlines links
% }
%-------------------------------------------------------------------------------
\section{About the Logo}
% \subsection{Author}
% Logo by Japan Lady in London for Japanese Photovoltaic Society
This Logo is created by Hitomi Terakawa of \href{hitomidesigns.com}{Hitomi Design Studio} based in London as a concept for the \href{https://www.j-pvs.jp/en/}{Japan Photovoltaic Society}. The green energy created by sunlight and wisdom. Powerful lines express powerful sunlight and sustainable power generation 
The dynamic circle represents the sun and its inexhaustible, ``eternal'' energy, with green evoking vitality, hope, and environmental awareness, and bright yellow evoking intelligence and a bright future.
The design also expresses the properties of light by using a method of mixing the three primary colors of light---the RGB color law---the composition, in which red light mixes with green and transforms into yellow, metaphorically represents the process of energy conversion. The primary color lines blend with the color of the central circle and change color according to the RGB color mixing law. This represents the dynamic sunlight (RGB lines) transforming into warm, ``green'' energy. The blue, red, and yellow lines represent the thoughts and wisdom of each researcher, and the green represents how that thought and wisdom is transformed into earth-friendly energy.
The design aims to attract attention and interest by creating a casual and cheerful atmosphere that is somewhat unrelated to images of chemistry, academic societies, and research.
This design metaphorically expresses the efforts of many people and the culmination of diverse academic research, making solar power generation possible. The three dynamic lines create a dignified impression within the bright image.
%-------------------------------------------------------------------------------
%===============================================================================
\end{document}
%===============================================================================



