%===============================================================================
% File Name     : <letterMaterichTemplate.tex>
% Description   : Materich Letter using <letterMaterich.cls>
%-------------------------------------------------------------------------------
% Author        : Amit Manohar Manthanwar
% Office        : Imperial College London
% Mailer        : amit@imperial.ac.uk
% Caller        : +44.207.594.6632
% Mobile        : +44.795.461.7111
% WebURL        : www.imperial.ac.uk
%-------------------------------------------------------------------------------
% Copyright     : ©2022 Amit Manohar Manthanwar
% License       : Restricted Confidential
%===============================================================================
%---------------+---------+----------------------------------------------------
% Revision Log  | Author  | Description
%---------------+---------+----------------------------------------------------
% 11-Sep-2022   | AMM     | Initial Version
%---------------+---------+----------------------------------------------------
%---------------+---------+----------------------------------------------------
%---------------+---------+----------------------------------------------------
%===============================================================================
% \documentclass{letterMaterich}
\documentclass{materichLetter}
%-------------------------------------------------------------------------------
%\title{\LaTeX{} Letter\thanks{}}
%\author{Amit M. Manthanwar}
%\date{\today}
%===============================================================================
\begin{document}
%-------------------------------------------------------------------------------
%\maketitle
\thispagestyle{letterhead}
\pagestyle{fancy}
%===============================================================================
\renewcommand{\fromName}{Sabina Ziemian}
\renewcommand{\fromRank}{PhD MBA}
\renewcommand{\fromDuty}{Chief Executive Officer}
\renewcommand{\fromDept}{Global Research Center}
\renewcommand{\fromPost}{Markt 29, 03226 Vetschau\\Brandenburg,
Germany}
\renewcommand{\fromFone}{+49.160.871.1056}
\renewcommand{\fromMail}{materich@outlook.de}
%\renewcommand{\fromSite}{\urlColor{teal}\href{http://materich.herokuapp.com/team/pwp?id=2}{materich.com}}
\renewcommand{\fromSite}{\urlColor{teal}\href{http://materich.com}{materich.com}}
\renewcommand{\fromSize}{36mm}
%-------------------------------------------------------------------------------
\renewcommand{\letterDate}{11 September 2022}
\renewcommand{\concerning}{
    \textsc{Partnership for EU Call
    \href{https://ec.europa.eu/info/funding-tenders/opportunities/portal/screen/opportunities/topic-details/horizon-eic-2022-pathfinderchallenges-01-01;callCode=null;freeTextSearchKeyword=water;matchWholeText=true;typeCodes=0,1,2,8;statusCodes=31094501,31094502;programmePeriod=null;programCcm2Id=null;programDivisionCode=null;focusAreaCode=null;destination=null;mission=null;geographicalZonesCode=null;programmeDivisionProspect=null;startDateLte=null;startDateGte=null;crossCuttingPriorityCode=null;cpvCode=null;performanceOfDelivery=null;sortQuery=sortStatus;orderBy=asc;onlyTenders=false;topicListKey=topicSearchTablePageState}
    {HORIZON-EIC-2022-PATHFINDERCHALLENGES-01-01}
    }
}
%-------------------------------------------------------------------------------
\renewcommand{\toName}{Professor Dr. Hubertus Pietsch}
\renewcommand{\toRank}{}
\renewcommand{\toDuty}{Head of Research}
\renewcommand{\toDept}{Bayer AG Pharmaceuticals Research and
Development \\MR and CT Contrast Media Research Department}
\renewcommand{\toPost}{Building S109, Müllerstraße 178 \\13353
Berlin, Germany}
\renewcommand{\toFone}{}
\renewcommand{\toMail}{}%hubertus.pietsch@bayer.com}


%===============================================================================
\printAddress
\par
%-------------------------------------------------------------------------------
\baselineskip = 15pt

%\fontfamily{ppl}\selectfont
%\fontfamily{phv}\selectfont
%\sffamily
%\renewcommand{\baselinestretch}{1}\normalsize

%===============================================================================
Dear Hubertus,

We are leading a collaborative multidisciplinary research project to
address the EU Pathfinder call on climate challenges for nitrogen
management and valorisation. Our research consortium is comprised of
established academic researchers from various European universities
and technology experts from around the world with a proven track
record of delivering successful research projects and technology
solutions. We are in a process of submitting our collaborative
research proposal to this EU call due on 19 October 2022 with an
expected budget of EUR 1.2 million.

Our proposed project aims to develop breakthrough materials, novel
processes, and integrated technology solutions for effectively
managing a variety of N-base molecules by converting them into
value-added products such as net zero commodities, chemicals, fuels,
and energy vectors. Our research underpins renewable energy as a
primary driver for innovation in N-conversion materials and
technologies by delivering the following work packages.

\begin{enumerate}
\item Material discovery for N-base interspecies conversion
\item Material optimisation for photocatalytic activities
\item System design, operation, and Industry 4.0 integration
\item Circular economy for sustainable process manufacturing
\item Knowledge and technology transfer to industry and academia
\end{enumerate}

Our novel materials and a suite of state-of-the-art process
technology solutions will address in an integrated manner:
environmental, industrial, agricultural, socioeconomic, and logistic
issues. We strongly believe that our research efforts will
significantly advance science and its key benefits to the industry.
Therefore we cordially invite you to join us in this undertaking of
high societal importance. You and your team can collaborate with us
in any number of the following areas.

\begin{enumerate}
\item Partnership in the proposed research project as an EU-funded
    or self-funded beneficiary
\item Voluntary in-kind assistance and/or any relevant cost-share
    commitments
\item Participation in the strategic planning and research
    development
\item Assistance with resources, scientific inputs, relevant
    datasets, and mentorship
\item Donation of new or used equipment, latest analytical tools,
    and software
\item Assistance in the promotion of research and dissemination of
    knowledge
\item Participation in curriculum development for internal and
    external audiences
\item Participation to promote transferable skills and vocational
    training
\item Commitment to the successful delivery of the proposed
    research project
\end{enumerate}

We believe that the research findings of this work may offer you some
of the following key benefits:
\begin{itemize}
\item Access to new knowledge in the area of nanotechnology
\item Ability to access material products and systems technologies
\item Opportunity to supervise and review the research findings
\item Opportunity to network with our subject matter experts
\item Opportunity to participate in prioritisation needs to advance
    the sector
\item Ability to gain in-depth knowledge of decision-making using
    the Industry 4.0 Platform
\item Access to on-site and online training for your current and
    future workforce
\end{itemize}
\par
\bigskip

I look forward to further discussing this EU call of action towards
establishing mutually beneficial ties with you to achieve our
commonly shared objectives of research innovation for industrial
success and environmental sustainability.

%===============================================================================
\par
\bigskip
%-------------------------------------------------------------------------------
Yours sincerely,\\[2mm]
Sabina
%-------------------------------------------------------------------------------
\bigskip
%===============================================================================
%\renewcommand{\refname}{Attachments:}
%-------------------------------------------------------------------------------
\begin{thebibliography}{99}
%===============================================================================
\setlength{\itemindent}{2em}
%-------------------------------------------------------------------------------
%-------------------------------------------------------------------------------
\bibitem{EU-Call-Website} Link to call:
    \href{https://ec.europa.eu/info/funding-tenders/opportunities/portal/screen/opportunities/topic-details/horizon-eic-2022-pathfinderchallenges-01-01;callCode=null;freeTextSearchKeyword=Pathfinder;matchWholeText=true;typeCodes=0,1,2,8;statusCodes=31094501,31094502,31094503;programmePeriod=null;programCcm2Id=null;programDivisionCode=null;focusAreaCode=null;destination=null;mission=null;geographicalZonesCode=null;programmeDivisionProspect=null;startDateLte=null;startDateGte=null;crossCuttingPriorityCode=null;cpvCode=null;performanceOfDelivery=null;sortQuery=sortStatus;orderBy=asc;onlyTenders=false;topicListKey=topicSearchTablePageState}
    {EU EIC 2022 pathfinder challenge}
%-------------------------------------------------------------------------------
\bibitem{EU-Challenge} Link to pdf:
    \href{https://eic.ec.europa.eu/document/download/eff5eba0-f117-4af7-867c-289046a0910a_en?filename=Challenge\%20guide_CO2-N_Management.pdf}
    {EU EIC 2022 pathfinder challenge guide on carbon dioxide and
    nitrogen}

%-------------------------------------------------------------------------------
\bibitem{EU-Call} Link to pdf:
    \href{https://eic.ec.europa.eu/document/download/8c9ca0e4-6d66-4d8c-be06-caa02b8d9d2c_en?filename=EIC-work-programme-2022-06-09.pdf}
    {EU EIC work programme}
%-------------------------------------------------------------------------------
\bibitem{Materich-pubs} Link to site:
    \href{https://materich.herokuapp.com/publications/}
    {Materich research work}
%===============================================================================
\end{thebibliography}
%===============================================================================
\end{document}
%===============================================================================

\begin{thebibliography}{9}
\bibitem[Doe]{doe} \emph{First and last \LaTeX{} example.},
John Doe 50 B.C.

\bibitem{doee} \emph{First and last \LaTeX{} example.},
John Doe 50 B.C.
