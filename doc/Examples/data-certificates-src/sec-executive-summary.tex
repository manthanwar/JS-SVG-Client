Commercial printing refers to a collection of services, such as layout
designing, binding, composition and press productions, used to transfer the
artwork and text onto paper and cards. The commercial printing process utilizes
a variety of materials such as flyers, brochures, books, posters, magazines,
newsletters, and transactional bills and statements. It plays an essential role in producing large displays, which aids in attracting consumers with attractive
designs. Consequently, it is widely used in the packaging, food and beverage,
pharmaceuticals, and publishing industries.


India represents one of the largest commercial printing markets in the Asia
Pacific region. The market is primarily driven by the development of innovative
printing technologies by the manufacturers. They have started focusing on
introducing engineered products with a reduced carbon footprint, higher energy
efficiency, and better resistance to chemicals such as solvents and cleaners. The market is further propelled by the use of commercial printing to its cost effectiveness and better print quality as compared to smaller printers. Apart from this, commercial printing is also crucial in the e-commerce industry, especially in the production of brochures, pamphlets and leaflets. Moreover, the transition to digital technology has also provided a positive impact on the market growth.

Printing and Print - Packaging industry in India is growing; people are
taking keen interest in this key industry now. There are more than 36 printing
institutes some of these giving even post-graduate education. Every year more
than 3500 new printing engineering graduates joins the industry, while still much more get on the spot training in the print shops. Printing especially Packaging printing is now one of the industry. It is said that since 1989 the growth of the Printing coupled with Packaging Printing industry is over 14\%. Today, India is fast becoming one of the major print producer \& manufacture of printed paper products for the world markets. The quality standards have improved dramatically and immense production capacities have been created. Some
printers have won recognition by winning prizes at international competition for
excellence in printing. The value of the print industry across India was over 225 billion Indian rupees in 2021. This was further expected to exceed 250 billion rupees by 2024, indicating a compound annual growth of around 3 percent.

The \textbf{Printing Cluster Tirupati} which is situated in the \textbf{City of Tirupati} is one such printing cluster which is providing services to the customers, industries and is an important part of the printing industrial sector in the region. The Cluster has \textbf{92 units employing more than 500 people directly} and nearly \textbf{3000 people indirectly} having a \textbf{turnover of Rs. 6440 Lakhs}.

In order to flourish in the ever changing market and trends the industry has to
take up necessary up gradations and has to embrace newer technologies to face
the market challenges. Majority of the micro industries are not able to embrace
newer technologies due to their financial constraints, as a result they are
suffering in terms of loss of market share, and reduced profit margins etc. In
order to overcome these kind of adversaries the Government can provide the
printing cluster units with necessary and strategic technological interventions in the form of Common Facility Center, which will be equipped with crucial
machineries which the cluster units can make use to correct and enhance their
production capabilities.

This Diagnostic Study is conducted by Intaglio Technical and Business
Services and Detailed Project Report (DPR) is prepared by IntaGlio Technical
and Business Service. The DSR and DPR suggest the hard interventions for the
cluster. The total project cost for the establishment of CFC is estimated as
Rupees 2519.26 Lakhs with GoI Grant of Rupees 1763.48 Lakhs, Govt. of
Andhra Pradesh Grant of Rupees 377.89 Lakhs and SPV contribution of Rupees
377.89 Lakhs. The Project is considered to be need based and support worthy.